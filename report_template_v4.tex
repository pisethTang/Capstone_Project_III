\documentclass[12pt,a4paper]{article}
% This text is inserted in the beginning of all
% LaTex and Tex files I create.
%
% File created: Tue Sep 26 2017
% File name:    report_template.tex
% Path:         /home/mroughan/Classes/AMPIII/Template/
%
% M Roughan
% Sept, 2017
%

% include a minimal set of useful packages
\usepackage{graphicx}
\usepackage{amsfonts} 
\usepackage{amssymb}
\usepackage{amsmath}
\usepackage[a4paper,top=3cm, bottom=3cm, left=3cm, right=3cm]{geometry}
\usepackage{lastpage}
\usepackage{fancyhdr}
\usepackage{url}

% PUT YOUR TITLE AND NAME HERE
\newcommand{\titlestr}{A new title \\ Final Reports}
\newcommand{\shorttitlestr}{A Template ...}
\newcommand{\authorstr}{A. Student} % INSERT YOUR NAME(S)

\begin{document}
%%%%%%%%%%%%%%%%%%%%%%%%%%%%%%555
% title page
\begin{titlepage}
  \centering
  
  {\LARGE \titlestr \par}

  \vspace{1cm}
  {\Large \authorstr \par}

  {\bf STUDENT NO.}

  \vspace{1cm}
  \today     % PUT YOUR DATE HERE

  \vspace{2cm}
  Report submitted for
  {\bf MATHS 3021}
  at the School of Mathematical Sciences,
  University of Adelaide

  \includegraphics[width=0.35\textwidth]{UoA_logo_col_vert.jpg}

  \vspace{2cm}
  \flushleft
  Project Area: {\bf LIST TOPIC AREA} \\
  Project Supervisor: {\bf INSERT NAME} \\

  \vspace{5mm} {\footnotesize In submitting this work I am indicating
    that I have read the University's Academic Integrity Policy. I
    declare that all material in this assessment is my own work except
    where there is clear acknowledgement and reference to the work of
    others.\par}

  \vspace{5mm} {\footnotesize I give permission for this work
    to be reproduced and submitted to other academic staff for
    educational purposes.\par}

  \vspace{5mm} {\footnotesize {\bf OPTIONAL:} I give permission this work
    to be reproduced and provided to future students as an exemplar report.\par}

  \vfill
\end{titlepage}

% put headings on each page
\pagestyle{fancy}
\fancyhf{}
\rhead{\shorttitlestr}
\lhead{\authorstr}
\rfoot{Page \thepage\ of \pageref{LastPage}}
\renewcommand{\headrulewidth}{1pt}

%%%%%%%%%%%%%%%%%%%%%%%%%%%%%%555
% abstract
\begin{abstract}
  This report is a simple template intended as a simple, consistent
  starting point for students to prepare  \LaTeX\ reports in the
  School of Mathematical Sciences at the University of Adelaide. 

  It is neither complete, nor perfect, but rather is aimed at making
  it easy for students to present a report which avoids many of the
  worst errors. 

  Included in each section are descriptions of how these might be
  written. However, mileage may vary. You should follow the advice of
  your lecturer in preference to any statement made here.
\end{abstract}


%\vspace{10mm}
%\noindent \hrulefill
%
%{\bf Notes:}
%{Your report should include a short summary (usually called an
%abstract). Discuss the role and style of abstract to be included with
%your supervisor or lecturer.
%
%Some general advice: this should be concise, but clear. If it is too
%long, it dilutes the important features, too short and it has no
%information. So it's a balancing act. What are the most important
%things someone reading your report should know about the task and the
%results?
%
%In an industry report it might often be called an ``executive
%summary,'' but in this case, it's even more crucial because it is
%often the only part your boss or customer will actually read! They
%don't want to wade through hundreds of pages of technical muck to get
%the message (they do want the muck -- it is needed to support the
%results -- but they might not examine it in detail).
%
%It's also the main draw card of an academic paper -- it's how I will
%decide whether to bother reading the rest of the article. \par}
%
%
%%%%%%%%%%%%%%%%%%%%%%%%%%%%%%%555
%% main report
%\clearpage
\section{Introduction}

The intention of this file is to provide a simple, consistent \LaTeX\
template for the School of Mathematical Sciences at the University of
Adelaide. Its goals are
\begin{enumerate}
\item to ease production of clean, appealing reports in \LaTeX ;

\item to allow us to provide consistent guidelines as to the desired
  length of the reports; and

\item to be a source of advice about writing your report.

\end{enumerate}
The intention is that students can replace text with their own, and so
start their report. 

The style and formatting here are minimal, so as to maintain a very
simple template, but there is no desire that students who wish to go
beyond this template should be restrained, as long as their final
report is consistent with quality and length requirements. In
particular, any alternative should respect the font size (12pt) and
the page size, which is A4, with 4cm margins to allow feedback.

This template also suggests a structure, but please replace section
headings with meaningful headings of your own. 

You may note that some pieces are omitted that you might have often
seen added, for instance, there is no Table of Contents. Many such things are added because of defunct ``rules''
rather than for utility. Here we have taken a minimalist style, unless a
specific feature serves a clear purpose. 

\noindent \hrulefill

{\bf Notes:}
An introduction must be strong, or your reader (with their limited
time), will give up on the work.

Important facets of an introduction are:
  \begin{itemize}
  \item introduce the basic ideas to be presented (at a high level);
  \item strongly motivate the work; 
  \item describe what you will do and present; and 
  \item give a summary of key results.
  \end{itemize}
 
You may notice it sounds a little like the abstract, and it
is. However, now you have space and time to go into more detail
(though it is still somewhat abstracted from the full detail).


\clearpage
\section{Background}

The first section, in many cases, that you present should present
related background material. In might include:
\begin{itemize}
\item literature review or related work section;
\item common notation and definitions; and/or
\item references for techniques to be used.
\end{itemize}

The type and detail of the content needed here depends strongly on the
audience. Some information may be omitted if it is common knowledge to
the specific audience, but care must be taken over any such
assumptions. Err on the conservative side.

\vspace{10mm}
\noindent \hrulefill

{\bf Notes:}
Writing a technical document is much like writing any other
document. There is still a story you are trying to tell. However,
there are certain features common to technical writing that you many
not have encountered.

In general, the goal of many technical reports is to convey more
precise, quantitative information than, for instance, a
novel. Technical writing should be approached by asking ``What does my
reader need to know so that they could reproduce my results exactly
(without asking me any supplemental questions)?'' 

The first starting point towards this goal is to define (i) any terms
or notation used precisely, (ii) provide a reader with definitive
references for techniques used, and (iii) to put the work into the
correct context within the larger scientific literature. 

Your introduction, or this section will be the first place you need to
include references.  BibTeX, and related tools are a superior means to
do so. There is one small example in this
template~\cite{guy17:_thing_i_wrote}.  Using references well is an
art. The approach varies depending on the use:
\begin{enumerate}

\item to allow a more concise description of a problem or method where
  it is already described in detail elsewhere;

\item to support arguments;

\item to give credit to other authors for their ideas or tools; and 

\item to provide links to additional information for the reader, for
  instance where to find a particular software package.

\end{enumerate}


\clearpage
\section{Methods}

The primary tool used in preparation of this report is \LaTeX , a
markup tool for the preparation of documents primarily used in
mathematics and related areas. 

Markup tools have the advantage of separating content from style, thus
allowing writers to focus on the content, and adding style (for
example, the format of section headings) later. It is a flexible and
powerful approach. 

The other key advantage of \LaTeX\ is the high quality of its
mathematical typesetting. There are no better tools for this task,
though there are many variants of \LaTeX\, and tools through which to
use it. 



\vspace{10mm}
\noindent \hrulefill

{\bf Notes:}
This template suggests that your next section should describe methods
used or developed in this report. Methods that are simple background
material should go in the previous section. This section focuses on
those that are novel, or in some cases just more difficult and more
important for the work.

Sometimes the section will be called ``methods,'' but a more
specific and descriptive heading is usually preferable. 

Your focus in describing these should be reproducibility. A reader
should ideally be able to recreate your work from your description. 

Describe data, experiments, simulations, or solution techniques such
that your reader can understand exactly what you did.  It may be
helpful to keep trying to answer the 6Ws: Why, When, Where, What, Who
and How.

However, the art of such writing is to balance detail and precision
with brevity. Concise descriptions are to be preferred because the
information is more accessible. Often we use references to allow us to
abbreviate or omit some details that are common to other experiments
or problems.

Mathematical notation is also very useful in composing precise, yet
concise descriptions of a problem.  However, do not use mathematics or
jargon for its own sake. Clarity is important, and mathematics or
complicated technical terms can either enhance this (when used
appropriately) or detract from it (if used carelessly). Your goal is
{\em not} to try to seem smart by using complicated words. Your goal is to
communicate!

A tutorial or examples of \LaTeX\ use are not included
here as there are now many sources of such information. For instance
see  \url{http://www.maths.adelaide.edu.au/anthony.roberts/LaTeX/index.html.}


\clearpage
\section{Results}
This template has no results to report. 

\vspace{10mm}
\noindent \hrulefill

{\bf Notes:} Remember the advice from the previous section. You need
your results to be concise, but once again be concrete, quantitative,
and provide enough information that the results could be reproduced
and verified. 

Figures and tables can be very useful. However, while a picture is
worth a thousand words, this is not true by itself. Any graph of
figure included in a report MUST have:
\begin{enumerate}
\item a detailed caption describing exactly what the figure shows
  (it should almost stand alone);
\item appropriate axes with labels including
  units; and 
\item discussion in the text of the document, not just the caption
  (make sure you refer to exact figure numbers in the text).
\end{enumerate}
Moreover, they should be easy to read with large enough text, and
clearly marked data points.  Tables should be treated similarly.  A
few such in a document are very useful, but be aware that deluging a
reading with figures and tables can be counter-productive. Part of the
art of technical writing is choosing good ways of informing the reader
of the critical information without diluting it with volumes of
irrelevancies.




\clearpage
\section{Conclusion}

This brief template is intended to provide a simple starting point for
students preparing \LaTeX\ reports.
 
It includes some advice about that report, but the brevity of this
report means that this advice is simplified and generic. You should
consult your lecturer for more detailed and specific advice.

\vspace{10mm}
\noindent \hrulefill

{\bf Notes:} All works should have a conclusion. Briefly summarise
your report (once again). Discuss the most important features of what
you have achieved, and the implications of your results.  The
conclusion should not introduce new information or ideas, however, if
you feel it is appropriate, you may speculate on directions for future
work.

\clearpage
\section*{Acknowledgements}

This template has grown out of earlier versions and so has had contributions from many
in the School of Mathematical Sciences including Liz Cousins, Yvonne Stokes, Ben Binder, Jono Tuke and Danny Stevenson.

\vspace{10mm}
\noindent \hrulefill

{\bf Notes:} It is common that you will want to acknowledge the
contribution of others to your work, even though these might not have
been sufficient to warrant being a co-author.

Consider who might have provided valuable discussions, funding
support, or moral support for the work. 

BTW, you don't have to start each section on a new page. It is done
here for clarity, but it isn't usually needed.

Also you should not have these notes section in your report.

\appendix

\section{Appendices}

This is a short appendix, just included as an example. 

\vspace{10mm}
\noindent \hrulefill

{\bf Notes:} 
An appendix can be used to include material that is important, but not
needed in the main body of the text, and which it might detract from
the main point of the report. 

A common example is code.  You should not include code in the main
body of a report unless it is particularly important or revealing.

However, for the convenience of your supervisors who may wish to
examine the code, and for your own benefit (in having a self-contained
document), you may wish to include the code in an appendix. If so,
have a look at the {\tt listings} package for \LaTeX. For Matlab,
there is also a {\tt matlab-prettifier} package that may work more
easily for you. 

\clearpage
\bibliographystyle{plain}
\begin{thebibliography}{9}
\bibitem{guy17:_thing_i_wrote}
Some Guy. A thing I wrote. \emph{The Journal of Stuff} {\bf 1}, 101--110, 2017.
\end{thebibliography}
%\bibliography{your_bib_files_name}

\vspace{10mm}
\noindent \hrulefill

{\bf Notes:} 
A critical component of the work is the list of references. We have
discussed their use earlier -- here some notes on their
presentation are simply given along with a pointer as to where more information can be found.

This is one of the hardest parts to get right. BibTeX can help a
great deal, although is not used here as it is more effort that it is worth for a single small report. Whether you use this or not, you need to take a good deal of care to make sure
that
\begin{itemize}
\item the references are in a consistent format;

\item all information is correct; and 

\item the information included is in the correct style for the
  intended audience.

\end{itemize}
There are a number of different referencing styles which differ in the way and order in which information is given. For information on different referencing styles see \url{https://www.adelaide.edu.au/library/referencing-support}. You are welcome to use whatever referencing style you prefer; just use it consistently. 
Details \emph{really} matter in this section and it's easy to lose marks.


\end{document}


 